\chapter{Specifikacija programske potpore}
		
	\section{Funkcionalni zahtjevi}
			
			\textbf{\textit{dio 1. revizije}}\\
			
			\textit{Navesti \textbf{dionike} koji imaju \textbf{interes u ovom sustavu} ili  \textbf{su nositelji odgovornosti}. To su prije svega korisnici, ali i administratori sustava, naručitelji, razvojni tim.}\\
				
			\textit{Navesti \textbf{aktore} koji izravno \textbf{koriste} ili \textbf{komuniciraju sa sustavom}. Oni mogu imati inicijatorsku ulogu, tj. započinju određene procese u sustavu ili samo sudioničku ulogu, tj. obavljaju određeni posao. Za svakog aktora navesti funkcionalne zahtjeve koji se na njega odnose.}\\
			
			
			\noindent \textbf{Dionici:}
			
			\begin{packed_enum}
				
				\item Dentalne klinike
				\item Vlasnici smještaja			
				\item Prijevoznici
				\item Administratori koji koriste aplikaciju
				\item Klijenti tj. pacijenti
				\item Razvojni tim
				
				
			\end{packed_enum}
			
			\noindent \textbf{Aktori i njihovi funkcionalni zahtjevi:}
			
			
			\begin{packed_enum}
				\item  \underbar{Smještajni administrator (inicijator) može:}
				
				\begin{packed_enum}
					
					\item Pregledavati podatke o smještajima (tip, ocjena, adresa, period dostupnosti)
					\item Dodavati nove korisnike (ne klijente nego korisnike sustava)
					\item Korisnicima dodavati nove uloge
					\item Dodati, izmjenjivati i brisati podatke o smještaju
					\item Vidjeti prikaz smještaja na karti
					
				\end{packed_enum}
				\eject
				
					\item  \underbar{Prijevoznički administrator (inicijator) može:}
				
				\begin{packed_enum}
					
					\item Pregledavati podatke o prijevoznicima
					\item Dodati osnovne osobne podatke prijevoznika
					\item Dodati kontaktne podatke i podatke o vrsti i kapacitetu vozila
					\item Izmjenjivati neosnovne podatke (kontakt, vrsta i kapacitet vozila)
					\item Izbrisati prijevoznika

				\end{packed_enum}
				
					\item  \underbar{Korisnički administrator (inicijator) može:}
				
				\begin{packed_enum}
					
					\item Dodati podatke o klijentima (osobni podatci, kontakt, preferencije o smještaju)
					
				\end{packed_enum}
			
				\item  \underbar{Baza podataka (sudionik) može:}
				
				\begin{packed_enum}
					
					\item Pohranjuje sve podatke o prijevoznicima
					\item Pohranjuje sve podatke o smještaju
					\item Pohranjuje ne medicinske podatke o klijentima
					
				\end{packed_enum}
				
					
			\end{packed_enum}
			
			\eject 
			
			
				
			\subsection{Obrasci uporabe}
				
				\textbf{\textit{dio 1. revizije}}
				
				\subsubsection{Opis obrazaca uporabe}
					\textit{Funkcionalne zahtjeve razraditi u obliku obrazaca uporabe. Svaki obrazac je potrebno razraditi prema donjem predlošku. Ukoliko u nekom koraku može doći do odstupanja, potrebno je to odstupanje opisati i po mogućnosti ponuditi rješenje kojim bi se tijek obrasca vratio na osnovni tijek.}\\
					

					\noindent \underbar{\textbf{UC$<$broj obrasca$>$ -$<$ime obrasca$>$}}
					\begin{packed_item}
	
						\item \textbf{Glavni sudionik: }$<$sudionik$>$
						\item  \textbf{Cilj:} $<$cilj$>$
						\item  \textbf{Sudionici:} $<$sudionici$>$
						\item  \textbf{Preduvjet:} $<$preduvjet$>$
						\item  \textbf{Opis osnovnog tijeka:}
						
						\item[] \begin{packed_enum}
	
							\item $<$opis korak jedan$>$
							\item $<$opis korak dva$>$
							\item $<$opis korak tri$>$
							\item $<$opis korak četiri$>$
							\item $<$opis korak pet$>$
						\end{packed_enum}
						
						\item  \textbf{Opis mogućih odstupanja:}
						
						\item[] \begin{packed_item}
	
							\item[2.a] $<$opis mogućeg scenarija odstupanja u koraku 2$>$
							\item[] \begin{packed_enum}
								
								\item $<$opis rješenja mogućeg scenarija korak 1$>$
								\item $<$opis rješenja mogućeg scenarija korak 2$>$
								
							\end{packed_enum}
							\item[2.b] $<$opis mogućeg scenarija odstupanja u koraku 2$>$
							\item[3.a] $<$opis mogućeg scenarija odstupanja  u koraku 3$>$
							
						\end{packed_item}
					\end{packed_item}
					
					\noindent \underbar{\textbf{UC1 - Prijava smještajnog administratora}}
					\begin{packed_item}
						
						\item \textbf{Glavni sudionik: }Smještajni administrator
						\item  \textbf{Cilj:} Prijaviti se u sustav kao smještajni administrator
						\item  \textbf{Sudionici:} Baza podataka
						\item  \textbf{Preduvjet:} Registracija?
						\item  \textbf{Opis osnovnog tijeka:}
						
						\item[] \begin{packed_enum}
							
							\item Unos korisničkog imena i lozinke te odabir uloge smještajnog administratora
							\item Potvrda o postojanju računa i ispravnosti podataka
							\item Pristup funkcijama smještajnog administratora
						\end{packed_enum}
						
						\item  \textbf{Opis mogućih odstupanja:}
						
						\item[] \begin{packed_item}
							
							\item[2.a] Uneseni podatci nisu u točnom formatu
							\item[] \begin{packed_enum}
								
								\item Obavijestiti korisnika koji podatci nisu u točnom formatu
								\item Korisnik mijenja potrebne podatke i pokušava opet
								
							\end{packed_enum}
							\item[2.b] Korisnički račun ne postoji, lozinka nije točna ili korisničko ime nije točno
							\item[] \begin{packed_enum}
								
								\item Obavijestiti korisnika da su korisničko ime ili lozinka netočni
								\item Korisnik mijenja potrebne podatke i pokušava opet
								
							\end{packed_enum}

							
						\end{packed_item}
					\end{packed_item}
				
				\noindent \underbar{\textbf{UC2 - Prijava prijevoznog administratora}}
				\begin{packed_item}
					
					\item \textbf{Glavni sudionik: }Prijevozni administrator
					\item  \textbf{Cilj:} Prijaviti se u sustav kao prijevozni administrator
					\item  \textbf{Sudionici:} Baza podataka
					\item  \textbf{Preduvjet:} Registracija?
					\item  \textbf{Opis osnovnog tijeka:}
					
					\item[] \begin{packed_enum}
						
						\item Unos korisničkog imena i lozinke te odabir uloge prijevoznog administratora
						\item Potvrda o postojanju računa i ispravnosti podataka
						\item Pristup funkcijama prijevoznog administratora
					\end{packed_enum}
					
					\item  \textbf{Opis mogućih odstupanja:}
					
					\item[] \begin{packed_item}
						
						\item[2.a] Uneseni podatci nisu u točnom formatu
						\item[] \begin{packed_enum}
							
							\item Obavijestiti korisnika koji podatci nisu u točnom formatu
							\item Korisnik mijenja potrebne podatke i pokušava opet
							
						\end{packed_enum}
						\item[2.b] Korisnički račun ne postoji, lozinka nije točna ili korisničko ime nije točno
						\item[] \begin{packed_enum}
							
							\item Obavijestiti korisnika da su korisničko ime ili lozinka netočni
							\item Korisnik mijenja potrebne podatke i pokušava opet
							
						\end{packed_enum}
						
						
					\end{packed_item}
				\end{packed_item}
				
				\noindent \underbar{\textbf{UC3 - Prijava korisničkog administratora}}
				\begin{packed_item}
					
					\item \textbf{Glavni sudionik: }Korisnički administrator
					\item  \textbf{Cilj:} Prijaviti se u sustav kao korisnički administrator
					\item  \textbf{Sudionici:} Baza podataka
					\item  \textbf{Preduvjet:} Registracija?
					\item  \textbf{Opis osnovnog tijeka:}
					
					\item[] \begin{packed_enum}
						
						\item Unos korisničkog imena i lozinke te odabir uloge korisničkog administratora
						\item Potvrda o postojanju računa i ispravnosti podataka
						\item Pristup funkcijama korisničkog administratora
					\end{packed_enum}
					
					\item  \textbf{Opis mogućih odstupanja:}
					
					\item[] \begin{packed_item}
						
						\item[2.a] Uneseni podatci nisu u točnom formatu
						\item[] \begin{packed_enum}
							
							\item Obavijestiti korisnika koji podatci nisu u točnom formatu
							\item Korisnik mijenja potrebne podatke i pokušava opet
							
						\end{packed_enum}
						\item[2.b] Korisnički račun ne postoji, lozinka nije točna ili korisničko ime nije točno
						\item[] \begin{packed_enum}
							
							\item Obavijestiti korisnika da su korisničko ime ili lozinka netočni
							\item Korisnik mijenja potrebne podatke i pokušava opet
							
						\end{packed_enum}
						
						
					\end{packed_item}
				\end{packed_item}
				
				\noindent \underbar{\textbf{UC4 - Dodavanje novog korisnika}}
				\begin{packed_item}
					
					\item \textbf{Glavni sudionik: }Smještajni administrator
					\item  \textbf{Cilj:} Dodati novog korisnika sustava
					\item  \textbf{Sudionici:} Baza podataka
					\item  \textbf{Preduvjet:} Prijava smještajnog administratora
					\item  \textbf{Opis osnovnog tijeka:}
					
					\item[] \begin{packed_enum}
						
						\item Unos korisničkog imena i lozinke novog korisnika te odabir svih uloga koje će korisnik imati
						\item Provjera postoji li već korisnik s takvim imenom
						\item Upis novog korisnika u bazu podataka
					\end{packed_enum}
					
					\item  \textbf{Opis mogućih odstupanja:}
					
					\item[] \begin{packed_item}
						
						\item[2.a] Polja korisničko ime i/ili lozinka su prazni
						\item[] \begin{packed_enum}
							
							\item Obavijestiti korisnika koji podatci su prazni
							\item Korisnik mijenja potrebne podatke i pokušava opet
							
						\end{packed_enum}
						\item[2.b] Postoji korisnički račun s tim imenom
						\item[] \begin{packed_enum}
							
							\item Obavijestiti korisnika da postoji račun s tim imenom
							\item Korisnik mijenja korisničko ime i pokušava opet
							
						\end{packed_enum}
						\item[2.c] Nije odabrana niti jedna uloga
						\item[] \begin{packed_enum}
							
							\item Obavijestiti korisnika da račun mora imati neku ulogu
							\item Korisnik dodaje jednu ili više uloga i pokušava opet
							
						\end{packed_enum}
						
					\end{packed_item}
				\end{packed_item}
				
				\noindent \underbar{\textbf{UC5 - Pregled postojećih smještaja}}
				\begin{packed_item}
					
					\item \textbf{Glavni sudionik: }Smještajni administrator
					\item  \textbf{Cilj:} Pregledati postojeće smještaje
					\item  \textbf{Sudionici:} Baza podataka
					\item  \textbf{Preduvjet:} Prijava smještajnog administratora
					\item  \textbf{Opis osnovnog tijeka:}
					
					\item[] \begin{packed_enum}
						
						\item Smještajni administrator odabire opcije "Postojeći smještaji"
						\item Aplikacija prikazuje kao listu sve postojeće smještaje
					\end{packed_enum}

				\end{packed_item}
				
				\noindent \underbar{\textbf{UC6 - Dodavanje novog smještaja}}
				\begin{packed_item}
					
					\item \textbf{Glavni sudionik: }Smještajni administrator
					\item  \textbf{Cilj:} Dodati novi smještaj
					\item  \textbf{Sudionici:} Baza podataka
					\item  \textbf{Preduvjet:} Prijava smještajnog administratora
					\item  \textbf{Opis osnovnog tijeka:}
					
					\item[] \begin{packed_enum}
						
						\item Unos potrebnih podataka za dodavanje novog smještaja
						\item Provjera jesu li uneseni svi podatci
						\item Upis novog smještaja u bazu podataka
					\end{packed_enum}
					
					\item  \textbf{Opis mogućih odstupanja:}
					
					\item[] \begin{packed_item}
						
						\item[2.a] Neki od podataka nije unesen
						\item[] \begin{packed_enum}
							
							\item Obavijestiti korisnika koji podatci su prazni
							\item Korisnik mijenja potrebne podatke i pokušava opet
							
						\end{packed_enum}
						
					\end{packed_item}
				\end{packed_item}
				
				\noindent \underbar{\textbf{UC7 - Promjena podataka smještaja}}
				\begin{packed_item}
					
					\item \textbf{Glavni sudionik: }Smještajni administrator
					\item  \textbf{Cilj:} Promijeniti podatke smještaja
					\item  \textbf{Sudionici:} Baza podataka
					\item  \textbf{Preduvjet:} Prijava smještajnog administratora
					\item  \textbf{Opis osnovnog tijeka:}
					
					\item[] \begin{packed_enum}
						
						\item Smještajni administrator odabire "Promijeni" opciju na smještaju
						\item Smještajni administrator mijenja podatke o smještaju
						\item Smještajni administrator sprema promijene
						\item Pregledava se ispravnost podataka
						\item Mijenjaju se podatci u bazi podataka
					\end{packed_enum}
					
					\item  \textbf{Opis mogućih odstupanja:}
					
					\item[] \begin{packed_item}
						
						\item[3.a] Smještajni administrator izađe bez spremanja
						\item[] \begin{packed_enum}
							
							\item Podatci se ne spremaju u bazu podataka
							
						\end{packed_enum}
						
						\item[4.a] Neki od podataka nije unesen
						\item[] \begin{packed_enum}
							
							\item Obavijestiti smještajnog administratora koji podatci su prazni
							\item Smještajni administrator mijenja potrebne podatke i pokušava opet
							
						\end{packed_enum}
						
					\end{packed_item}
				\end{packed_item}
				
				\noindent \underbar{\textbf{UC8 - Brisanje smještaja}}
				\begin{packed_item}
					
					\item \textbf{Glavni sudionik: }Smještajni administrator
					\item  \textbf{Cilj:} Promijeniti podatke smještaja
					\item  \textbf{Sudionici:} Baza podataka
					\item  \textbf{Preduvjet:} Prijava smještajnog administratora
					\item  \textbf{Opis osnovnog tijeka:}
					
					\item[] \begin{packed_enum}
						
						\item Smještajni administrator odabire "Izbriši" opciju na smještaju
						\item Aplikacija pokazuje upit smještajnom administratoru je li siguran
						\item Ako je smještaj se briše iz baze u protivnom se ništa ne desi
					\end{packed_enum}

				\end{packed_item}
				
			\noindent \underbar{\textbf{UC9 - Pregled postojećih prijevoznika}}
			\begin{packed_item}
				
				\item \textbf{Glavni sudionik: }Prijevozni administrator
				\item  \textbf{Cilj:} Pregledati postojeće prijevoznike
				\item  \textbf{Sudionici:} Baza podataka
				\item  \textbf{Preduvjet:} Prijava prijevoznog administratora
				\item  \textbf{Opis osnovnog tijeka:}
				
				\item[] \begin{packed_enum}
					
					\item Prijevozni administrator odabire opcije "Postojeći prijevoznici"
					\item Aplikacija prikazuje kao listu sve postojeće prijevoznike
				\end{packed_enum}
				
			\end{packed_item}
			
			\noindent \underbar{\textbf{UC10 - Dodavanje novog prijevoznika}}
			\begin{packed_item}
				
				\item \textbf{Glavni sudionik: }Prijevozni administrator
				\item  \textbf{Cilj:} Dodati novog prijevoznika
				\item  \textbf{Sudionici:} Baza podataka
				\item  \textbf{Preduvjet:} Prijava prijevoznog administratora
				\item  \textbf{Opis osnovnog tijeka:}
				
				\item[] \begin{packed_enum}
					
					\item Unos potrebnih podataka za dodavanje novog prijevoznika
					\item Provjera jesu li uneseni svi podatci
					\item Upis novog prijevoznika u bazu podataka
				\end{packed_enum}
				
				\item  \textbf{Opis mogućih odstupanja:}
				
				\item[] \begin{packed_item}
					
					\item[2.a] Neki od podataka nije unesen
					\item[] \begin{packed_enum}
						
						\item Obavijestiti prijevoznog administratora koji podatci su prazni
						\item Prijevozni administrator mijenja potrebne podatke i pokušava opet
						
					\end{packed_enum}
					
				\end{packed_item}
			\end{packed_item}
			
			\noindent \underbar{\textbf{UC11 - Promjena podataka o prijevozniku}}
			\begin{packed_item}
				
				\item \textbf{Glavni sudionik: }Prijevozni administrator
				\item  \textbf{Cilj:} Promijeniti podatke o prijevozniku
				\item  \textbf{Sudionici:} Baza podataka
				\item  \textbf{Preduvjet:} Prijava prijevoznog administratora
				\item  \textbf{Opis osnovnog tijeka:}
				
				\item[] \begin{packed_enum}
					
					\item Prijevozni administrator odabire "Promijeni" opciju na postojećem prijevozniku
					\item Prijevozni administrator mijenja podatke o prijevozniku
					\item Prijevozni administrator sprema promijene
					\item Pregledava se ispravnost podataka
					\item Mijenjaju se podatci u bazi podataka
				\end{packed_enum}
				
				\item  \textbf{Opis mogućih odstupanja:}
				
				\item[] \begin{packed_item}
					
					\item[3.a] Prijevozni administrator izađe bez spremanja
					\item[] \begin{packed_enum}
						
						\item Podatci se ne spremaju u bazu podataka
						
					\end{packed_enum}
					
					\item[4.a] Neki od podataka nije unesen
					\item[] \begin{packed_enum}
						
						\item Obavijestiti prijevoznog administratora koji podatci su prazni
						\item Prijevozni administrator mijenja potrebne podatke i pokušava opet
						
					\end{packed_enum}
					
				\end{packed_item}
			\end{packed_item}
			
			\noindent \underbar{\textbf{UC12 - Brisanje prijevoznika}}
			\begin{packed_item}
				
				\item \textbf{Glavni sudionik: }Prijevozni administrator
				\item  \textbf{Cilj:} Promijeniti podatke smještaja
				\item  \textbf{Sudionici:} Baza podataka
				\item  \textbf{Preduvjet:} Prijava prijevoznog administratora
				\item  \textbf{Opis osnovnog tijeka:}
				
				\item[] \begin{packed_enum}
					
					\item Prijevozni administrator odabire "Izbriši" opciju na prijevozniku
					\item Aplikacija pokazuje upit prijevoznom administratoru je li siguran
					\item Ako je, prijevoznik se briše iz baze u protivnom se ništa ne desi
				\end{packed_enum}
				
			\end{packed_item}
			
			\noindent \underbar{\textbf{UC13 - Dodavanje novog klijenta}}
			\begin{packed_item}
				
				\item \textbf{Glavni sudionik: }Korisnički administrator
				\item  \textbf{Cilj:} Dodati novog korisnika
				\item  \textbf{Sudionici:} Baza podataka
				\item  \textbf{Preduvjet:} Prijava korisničkog administratora
				\item  \textbf{Opis osnovnog tijeka:}
				
				\item[] \begin{packed_enum}
					
					\item Unos potrebnih podataka za dodavanje novog korisnika
					\item Dohvat medicinskih podataka o korisniku
					\item Provjera jesu li uneseni svi podatci
					\item Dodjela smještaja klijentu
					\item Slanje poruke elektroničke pošte klijentu i prijevozniku o zaključenom planu 

				\end{packed_enum}
				
				\item  \textbf{Opis mogućih odstupanja:}
				
				\item[] \begin{packed_item}
					
					\item[2.b] Ne mogu se dohvatiti medicinski podatci
					\item[] \begin{packed_enum}
						
						\item Aplikacija pokušava dohvatiti podatke opet
						\item Ako je podatke nemoguće dohvatiti, nemoguće je unijeti novog korisnika
						
					\end{packed_enum}
					
					\item[3.a] Neki od podataka nije unesen
					\item[] \begin{packed_enum}
						
						\item Obavijestiti korisničkog administratora koji podatci su prazni
						\item Korisnički administrator mijenja potrebne podatke i pokušava opet
						
					\end{packed_enum}
					
				\end{packed_item}
			\end{packed_item}
			
					
				\subsubsection{Dijagrami obrazaca uporabe}
					
					\textit{Prikazati odnos aktora i obrazaca uporabe odgovarajućim UML dijagramom. Nije nužno nacrtati sve na jednom dijagramu. Modelirati po razinama apstrakcije i skupovima srodnih funkcionalnosti.}
				\eject		
				
			\subsection{Sekvencijski dijagrami}
			\textbf{Obrazac uporabe UC4: Dodavanje novog korisnika}
			
			Smještajni administrator odabire opciju dodaj novog korisnika. Web-aplikacija mu otvara obrazac za dodaju novog korisnika. Smještajni administrator upisuje korisničko ime, lozinku i odabire uloge koje će novi administrator imati i šalje zahtjev aplikaciji. Aplikacija radi provjeru ispravnosti tih podataka i ako su ispravni šalje ih bazi podataka. Baza podataka provjerava postoji li već administrator s tim korisničkim imenom i ako ne dodaje novog administratora. Smještajnom administratoru aplikacija javlja da je unos uspješno izvršen.
			
				
				\textbf{\textit{dio 1. revizije}}\\
				
				\textit{Nacrtati sekvencijske dijagrame koji modeliraju najvažnije dijelove sustava (max. 4 dijagrama). Ukoliko postoji nedoumica oko odabira, razjasniti s asistentom. Uz svaki dijagram napisati detaljni opis dijagrama.}
				\eject
	
		\section{Ostali zahtjevi}
		
			\textbf{\textit{dio 1. revizije}}\\
		 
			 \textit{Nefunkcionalni zahtjevi i zahtjevi domene primjene dopunjuju funkcionalne zahtjeve. Oni opisuju \textbf{kako se sustav treba ponašati} i koja \textbf{ograničenja} treba poštivati (performanse, korisničko iskustvo, pouzdanost, standardi kvalitete, sigurnost...). Primjeri takvih zahtjeva u Vašem projektu mogu biti: podržani jezici korisničkog sučelja, vrijeme odziva, najveći mogući podržani broj korisnika, podržane web/mobilne platforme, razina zaštite (protokoli komunikacije, kriptiranje...)... Svaki takav zahtjev potrebno je navesti u jednoj ili dvije rečenice.}
			 
			 
			 
	